\chapter{Introdução}
Neste capítulo encontram-se diversas definições e conceitos necessários à compreensão da cadeira de Elementos de Probabilidade e Estatística.

\section{Experiência}

\subsection{Experiência aleatória (e.a.)}
Trata-se de uma \textbf{experiência aleatória (e.a.)} se puder conduzir a dois ou mais resultados.

\subsection{Experiência determinista}
Trata-se de uma \textbf{experiência determinista} se apenas conduz a um resultado possível.

\subsection{Espaço de resultados de uma experiência aleatória}
Conjunto de todos os resultados possíveis.
Representa-se por S ou $\Omega$.
\begin{itemize}
    \item Contável
    \item Não contável
\end{itemize}

\section{Acontecimento}
Qualquer subconjunto do espaço de resultados.

\subsection{Acontecimento elementar}
Existe apenas um caso do universo de resultados que lhe é favorável.

\subsection{Acontecimento composto}
O número de casos favoráveis é suerior a um.

\section{Probabilidade}

\subsection{Definição clássica de probabilidade}
\begin{itemize}
    \item Dados $\Omega$, espaço de resultados, e A, acontecimento: $A \subseteq \Omega$
    \item Dado $\Omega$ finito, os resultados sejam todos equiprováveis.
    \item $P(A) = \frac{\# A}{\# \Omega} = \frac{\text{Número de resultados favoráveis a A}}{\text{Número de resultados possíveis}}$
    \item Só permite avaliar experiências repetíveis e em condições iguais.
    \item Tem de ser possível repetir a experiência aleatória tantas vezes quantas as que se queira.
    \item A experiência aleatória tem de ser repetida em condições sempre iguais.
\end{itemize}

\subsection{Axiomática de Kolmogorov}
\begin{enumerate}
    \item $P(\Omega) = 1$
    \item $\forall A \subseteq \Omega$, $P(A \geq 1)$
    \item $A_1, A_2, ..., A_n: A_i \cap A_j = \emptyset \Rightarrow P (\bigcup\limits_{k \geq 1} A_k) = \sum\limits_{k\geq 1} P(A_k)$
\end{enumerate}
\begin{enumerate}[I)]
    \item .
    \item .
    \item .
    \item .
\end{enumerate}