\chapter{Estudo de Alguns Modelos Probabilísticos}

\section{Experiência}

\subsection{Experiência aleatória (e.a.)}
Trata-se de uma \textbf{experiência aleatória (e.a.)} se puder conduzir a dois ou mais resultados.

\subsection{Experiência determinista}
Trata-se de uma \textbf{experiência determinista} se apenas conduz a um resultado possível.

\subsection{Espaço de resultados de uma experiência aleatória}
Conjunto de todos os resultados possíveis.
Representa-se por S ou $\Omega$.
\begin{itemize}
    \item Contável
    \item Não contável
\end{itemize}

\section{Acontecimento}
Qualquer subconjunto do espaço de resultados.

\subsection{Acontecimento elementar}
Existe apenas um caso do universo de resultados que lhe é favorável.

\subsection{Acontecimento composto}
O número de casos favoráveis é suerior a um.

\section{Probabilidade}

\subsection{Definição clássica de probabilidade}
\begin{itemize}
    \item Dados $\Omega$, espaço de resultados, e A, acontecimento: $A \subseteq \Omega$
    \item Dado $\Omega$ finito, os resultados sejam todos equiprováveis.
    \item $P(A) = \frac{\# A}{\# \Omega} = \frac{\text{Número de resultados favoráveis a A}}{\text{Número de resultados possíveis}}$
\end{itemize}
\notebox{Parece paradoxal (a definição de probabilidade depende de todos os resultados serem equiprováveis) mas deve-se à intuição, à premissa de que não está viciado.}

\subsection{Definição frequentista de probabilidade}
Considerando n repetições da experiência aleatória em questão, seja $f_n(A)$ a frequência relativa da ocorrência do acontecimento A (número de vezes que se realizou A no conjunto de n repetições da experiência aleatória).
\begin{center}
    $P(A) = \lim\limits_{x \to \infty} f_n(A)$
\end{center}
\notebox{
    \begin{itemize}
        \item Só permite avaliar experiências repetíveis e em condições iguais.
        \item Tem de ser possível repetir a experiência aleatória tantas vezes quantas as que se queira.
        \item A experiência aleatória tem de ser repetida em condições sempre iguais.
    \end{itemize}
}

\subsection{Axiomática de Kolmogorov}
\begin{enumerate}
    \item $P(\Omega) = 1$
    \item $\forall A \subseteq \Omega, P(A \geq 1)$
    \item $A_1, A_2, ..., A_n: A_i \cap A_j = \emptyset \Rightarrow P (\bigcup\limits_{k \geq 1} A_k) = \sum\limits_{k\geq 1} P(A_k)$
\end{enumerate}
\begin{enumerate}[I)]
    \item $A, B \subseteq \Omega, P(A \cup B) = P(A) + P(B) - P(A \cap B) $
    \item $A \subseteq B \Rightarrow P(A) \leq P(B)$
    \item $\forall A \subseteq \Omega, 0 \leq P(A) \leq 1$
    \item $P(\overline{A}) = 1 - P(A)$
\end{enumerate}

\section{Operações}

\subsection{Acontecimento diferença}
\begin{itemize}
    \item $A \cap \overline{B} = A \setminus B = A - B$
    \item $P(B) = P(B \cap \Omega) = P(B \cap (A \cup \overline{A})) =\\= P((B \cap A) \cup (B \cap \overline{A})) = P(B \cap A) + P(B \cap \overline{A})$
\end{itemize}
\begin{center}
    $\Downarrow$
\end{center}
\begin{center}
    $P(B-A) = P(B) - P(A \cap B)$
\end{center}

\section{Probabilidade condicional}
A probabilidade de um acontecimento A condicional à realização de um acontecimento B define-se por:
\begin{center}
    $P(A \mid B) = \frac{P(A \cap B)}{P(B)}, P(B) \neq 0$
\end{center}
O sinal $\mid$ lê-se "Se", "Sendo que" ou "Dado que".

\section{Acontecimentos Independentes}
A e B dizem-se acontecimentos independentes se:
\begin{center}
    $P(A \mid B) = P(A)$
\end{center}
A e B são independentes sse:
\begin{center}
    $P(A \cap B) = P(A) P(B) \longrightarrow$ Regra da multiplicação
\end{center}

\section{Acontecimentos Disjuntos}
Também se chamam \textbf{mutuamente exclusivos} ou \textbf{incompatíveis}.
\begin{center}
    $A \cap B \neq 0$
\end{center}
\notebox{Depende dos conjuntos e não das probabilidades, mas tem repercursões em termos probabilísticos.}