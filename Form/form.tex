\documentclass{article}

\usepackage[utf8]{inputenc}
\usepackage{graphicx}
\usepackage{indentfirst}
\usepackage[portuguese]{babel}
\usepackage{amsmath}
\usepackage{enumerate}

\geometry{
    a4paper,
    right=5mm,
    left=5mm,
    top=2.5mm,
    bottom=15mm,
    }

\begin{document}

\begin{center}
    \begin{table}
        \centering
        \bgroup
        \def\arraystretch{2}
        \begin{tabular}{||l||l||L|L|L|L|L||} 
        \hhline{~~||=|=|=|=|=||}
        \multicolumn{1}{l}{}       &                  & \text{Notação}                                                      & E(X)                  & Var(X)                    & P(X=K)                                                                    & P(X < K)  \\ 
        \hline\hline
        \multirow{5}{*}{Discretas} & Bernoulli        & X\frown B(p)                                                        & p                     & p(1-p)                    & -                                                                         & -                      \\ 
        \cline{2-7}
                                   & Binomial         & X\frown Bi(n,p)                                                     & np                    & np(1-p)                   & {n \choose k} p^k(1-p)^{n-k}                                              & -                 \\ 
        \cline{2-7}
                                   & Geométrica       & X\frown Geom()                                                      &   -                   &\frac{1-p}{p^2}            &  -                                                                        &  -                        \\ 
        \cline{2-7}
                                   & Hipergeométrica  & X\frown H(N,K,n)                                                    & \frac{nK}{N}          & -                         & -                                                                         &  -                    \\ 
        \cline{2-7}
                                   & Poisson          & X\frown Poisson(\lambda)                                            &  \lambda              &   \lambda                 & e^{-\lambda}\frac{\lambda^k}{k!}                                          &         -          \\ 
        \hline\hline
        \multirow{5}{*}{Continuas} & Uniforme         & X\frown U[a,b]                                                      & \frac{a+b}{2}         &\frac{(b-a)^2}{12}         & \frac{1}{b-a}                                                             &  \frac{x-a}{b-a},  a\leq x<b                 \\ 
        \cline{2-7}
                                   & Exponencial      & X\frown Exp(\lambda)                                                & \frac{1}{\lambda}     & \frac{1}{\lambda ^2}      &  \lambda e^{-\lambda x}, x \geq 0                                         &  1-\lambda e^{-\lambda x}, x \geq 0                     \\ 
        \cline{2-7}
                                   & Gaussiana/Normal & X\frown N(\mu, \sigma) \text{ou} X\frown N(\mu, \sigma ^2)          &  \mu                  &  \sigma ^2                &  \frac{e^{-\frac{1}{2}(\frac{x-\mu}{\sigma})^2}}{\sqrt{e\pi} \sigma}      &  \Phi                     \\ 
        \cline{2-7}
                                   & Qui-Quadrado     & \Phi \frown \chi _{k}^{2}                                           &  k                    &  2k                       &   -                                                                       &  -                     \\ 
        \cline{2-7}
                                   & t de Student     &  -                                                                  &  0                    &  -                        &   -                                                                       &  -                       \\
        \hline\hline
        \end{tabular}
        \egroup
        \end{table}
\end{center}

\begin{multicols*}{2}
\begin{itemize}
    \item Normalização
    \begin{itemize}
        \item Teorema do limite central
        \begin{itemize}
            \item Sejam $X_1, ..., X_n$ variáveis aleatórias independentes e identicamente distribuídas, com variância finita, tais que $E(X_i)=\mu$ e $Var(X)=\sigma^2$, $\forall i=1,...,n$.
            \item Seja $S_n = \sum_{i=1}^nX_i$.
            \item $$P(\frac{S_n-n\mu}{\sigma\sqrt{n}}\leq y) \longrightarrow \Phi(y)$$ 
            \item $$P(\frac{\bar{X}-\mu}{\sigma\sqrt{n}}\leq y) \longrightarrow \Phi(y)$$ 
        \end{itemize}
        \item Normal padrão
        \begin{itemize}
            \item $$X\frown N(\mu, \sigma^2) \Rightarrow Z=\frac{X-\mu}{\sigma}\frown N(0,1)$$
            \item $$Z\frown N(0, 1) \Rightarrow X=\sigma Z+\mu\frown N(\mu,\sigma^2)$$
        \end{itemize}
        \item Normalização do Qui-Quadrado
        \begin{itemize}
            \item Se $(X_1, ..., X_n)$ é uma a.a. de $X\frown N(\mu, \sigma)$:
            \item $$\frac{(n-1)S^2}{\sigma^2}\frown \chi_{n-1}^2, S^2=\frac{1}{n-1}\sum_{i=1}^n(X_i-\bar{X})^2$$
            \item $$\bar{X}\frown N(\mu, \frac{\sigma}{\sqrt{n}}), \bar{X}=\frac{1}{n}\sum_{i=1}^nX_i$$
        \end{itemize}
        \item Normalização do t-Student
        \begin{itemize}
            \item $$T=\frac{Z}{\sqrt{\frac{Q}{K}}}=\frac{\frac{\bar{X}-\mu}{\frac{\sigma^2}{\sqrt{n}}}}{\sqrt{\frac{\frac{(n-1)S^2}{\sigma^2}}{n-1}}}$$
            \item $$T=\frac{\bar{X}-\mu}{S}\sqrt{n}\frown t(n-1)$$
        \end{itemize}
    \end{itemize}

    \vfill\null
    \columnbreak

    \item Intervalos de confiança
    \begin{itemize}
        \item População Normal
        \begin{itemize}
            \item Estimar E(X)
            \begin{itemize}
                \item $\sigma$ conhecido:
                $$IC_{(1- \alpha)100\%}(\mu)=(\bar{X}\mp \frac{\sigma}{\sqrt{n}}z_{1-\frac{\alpha}{2}})$$
                \item $\sigma$ desconhecido
                $$IC_{(1- \alpha)100\%}(\mu)=(\bar{X}\mp \frac{S}{\sqrt{n}}t_{1-\frac{\alpha}{2};n-1})$$
            \end{itemize}
            \item Estimar $\sigma$
            \begin{itemize}
                \item $\mu$ conhecido
                $$IC_{(1- \alpha)100\%}(\sigma^2)=(\frac{(n-1)S^2}{\chi_{1-\frac{\alpha}{2};n}},\frac{(n-1)S^2}{\chi_{\frac{\alpha}{2};n}})$$
                \item $\mu$ desconhecido
                $$IC_{(1- \alpha)100\%}(\sigma^2)=$$$$=(\frac{\sum_{i=1}^n(X_i-\mu)^2}{\chi_{1-\frac{\alpha}{2};n}},\frac{\sum_{i=1}^n(X_i-\mu)^2}{\chi_{\frac{\alpha}{2};n}})$$
            \end{itemize}
        \end{itemize}
        \item População p
        \begin{itemize}
            \item 1 População
            $$IC_{(1- \alpha)100\%}(p)=(\hat{p}\mp z_{1-\frac{\alpha}{2}}\sqrt{\frac{\hat{p}(1-\hat{p})}{n}})$$
            \item 2 Populações
            \begin{itemize}
                \item $\sigma_1^2, \sigma_2^2$ conhecidos
                $$\mu_1, \mu_2=(\bar{X}_1-\bar{X}_2\mp z_{1-\frac{\alpha}{2}}\sqrt{\frac{\sigma_1^2}{n_1}+\frac{\sigma_2^2}{n_2}})$$
                \item $\sigma_1^2 = \sigma_2^2$ desconhecidos
                $$\mu_1-\mu_2=(\bar{X}_1-\bar{X}_2\mp t_{n_1+n_2-2;1-\frac{\alpha}{2}}\sqrt{S_p^2(\frac{1}{n_1}+\frac{1}{n_2})})$$
            \end{itemize}
        \end{itemize}
    \end{itemize}
\end{itemize}
\end{multicols*}







\newpage
\begin{multicols*}{2}
\begin{itemize}
    \item{Axiomática de Kolmogorov}
        \begin{enumerate}
            \item $P(\Omega) = 1$
            \item $\forall A \subseteq \Omega, P(A \geq 1)$
            \item $A_1, A_2, ..., A_n: A_i \cap A_j = \emptyset \Rightarrow P (\bigcup\limits_{k \geq 1} A_k) = \sum\limits_{k\geq 1} P(A_k)$
        \end{enumerate}
        \begin{enumerate}[I)]
            \item $A, B \subseteq \Omega, P(A \cup B) = P(A) + P(B) - P(A \cap B) $
            \item $A \subseteq B \Rightarrow P(A) \leq P(B)$
            \item $\forall A \subseteq \Omega, 0 \leq P(A) \leq 1$
            \item $P(\overline{A}) = 1 - P(A)$
        \end{enumerate}

    \item{Probabilidade condiciomal}
        $$P(A \mid B) = \frac{P(A \cap B)}{P(B)}, P(B) \neq 0$$

    \item{Teorema de Bayes}
        $$P(C_i \mid B) = \frac{P(B \mid C_i)P(C_i)}{\sum_{j=1}^{\infty} P(B \mid C_j)P(C_J)}$$

    \item{Funções estatísticas}
    \begin{itemize}
        \item Valor esperado
        \begin{itemize}
            \item $E[x] = \sum\limits_{i} x \cdot P(X=x_i)$
            \item $E[aX + b] = aE[x] + b$
            \item $E[aX] = aE[x]$
        \end{itemize}
        \item Variância
        \begin{itemize}
            \item $\sigma^2 = \text{Var}[x] = E[(x-E(x))^2] = E[(X-\mu)^2] = E(x^2) - \mu^2 = E(x^2) - (E(x))^2 $
            \item $ \text{Var}[x] \geq 0 $
            \item $\text{Var}[aX + b] = a^2\text{Var}[x] $
            \item $\text{Var}[X] + b = \text{Var}[x] $
            \item $\text{Var}[b] = 0 $
            \item $\text{Var}[X + Y] = \text{Var}[X] + \text{Var}[Y]$
        \end{itemize}
        \item Desvio padrão
        \begin{itemize}
            \item $\sigma (X) = \sqrt{\text{Var}(X)}$
        \end{itemize}
    \end{itemize}

    \item{Variáveis independentes}
    \begin{itemize}
        \item X e Y dizem-se independentes sse $F_{(X,Y)}(x,y)=P(X\leq x, Y\leq y)=F_X(x)\times F_Y(y)$
        \item Distribuição Binomial
        \begin{itemize}
            \item $X+Y\frown Bi(n_X+n_Y, p)$
        \end{itemize}
        \item Distribuição Poisson
        \begin{itemize}
            \item $X+Y\frown Poisson(\lambda + \mu)$
        \end{itemize}
        \item Distribuição Normal
        \begin{itemize}
            \item $X+Y\frown N(\mu_X+\mu_Y, \sqrt{\sigma_X^2+\sigma_Y^2})$
        \end{itemize}
    \end{itemize}

    \item{Pares de variáveis discretas}
    \begin{itemize}
        \item $l_{X,Y}=\frac{cov(X,Y)}{\sigma_X \sigma_Y}$
        \item $cov(X,Y)=E(XY)-E(X)E(Y)$
        \item $\text{X,Y independentes} \rightarrow \newline \rightarrow P(X=x_i)P(Y=y_j), \forall(x_i, y_j)$
    \end{itemize}    

\end{itemize}
\end{multicols*}

\end{document}